\documentclass[12pt]{article}
\usepackage[utf8]{inputenc}
\usepackage{amsmath}
\usepackage{amsfonts}
\usepackage{amsthm}
\usepackage{bm}
\usepackage{amssymb}
\usepackage{graphicx}
\usepackage{fullpage}

\newcommand{\true}{\text{vrai}}
\newcommand{\false}{\text{faux}}

%%%%%%%%%%%%%%%%%%%%%%%%%%%%%%%%%%%%%%%%%%%%%%%%%%%%%%%%%

\title{LU3IN003 - PROJET \\ Un problème de tomographie discrète\\}
\author{Esther CHOI (3800370) et Vinh-Son PHO ()}

\begin{document}
	\maketitle
	\newpage
	
	\section{Méthode incomplète de résolution}
	
		\subsection{Première étape}
		
			\paragraph{Q1}
				Il suffit de regarder s'il existe $ j \in \{1,...,M-1\} $ tel que $ T(j,k) = \true $. En effet cela signifierait qu'il existe un coloriage possible des $ j+1 $ premières cases avec la séquence complète $ (s_1,...,s_k) $.
		
\end{document}